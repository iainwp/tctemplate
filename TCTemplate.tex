\documentclass{slides}

%Condes PDFs are v1.7
\pdfminorversion=7

\usepackage[a4paper,portrait,margin=1cm]{geometry}
\usepackage{graphicx}
\usepackage{tikz}
\usepackage{rotating}
\usepackage{anyfontsize}
\usepackage{forloop}

%Set font to TeX Gyre Heros - very similar to Helvetica
\renewcommand{\familydefault}{qhv}

\def\AdjustMode{0}
\newcommand{\NumStations}{2}
\newcommand{\MaxProblemsPerStation}{4}
\def\ProblemsPerStationList{{2,4}}
\def\StationIDList{{"1","2"}}
\def\NumKitesList{{2,3}}
\def\ZeroOptionList{{1,1}}
\def\MapHeadingList{{3,34}}
\def\SquareMapList{{0,1}}
\def\CircleRadiusList{{2.5,6}}
\def\BriefingWidthList{{"3.5cm","8.399999999999998cm",}}
\def\MapScaleList{{5000,5000}}
\def\ContourIntervalList{{5,5}}
\def\MapFileList{{{"TempOTest","TempOTest"},{"TempOTest","TempOTest","TempOTest","TempOTest"}}}
\def\MapPageList{{{2,3},{4,5,6,7}}}
\def\ControlxCoordinateList{{{18.2141525,20.6028915},{18.7202407,18.2141525,21.277676300000003,20.6028915}}}
\def\ControlyCoordinateList{{{21.057652,20.544815100000004},{19.478656,21.050902,22.420718400000002,20.544815100000004}}}
\def\DescriptionFileList{{{"CDs","CDs"},{"CDs","CDs","CDs","CDs"}}}
\def\DescriptionPageList{{{2,3},{4,5,6,7}}}
\def\DescriptionxCoordinateList{{{1.25,1.25},{1.25,1.25,1.25,1.25}}}
\def\DescriptionyCoordinateList{{{26.28,25.580000000000002},{25.580000000000002,25.580000000000002,25.580000000000002,25.580000000000002}}}
\def\DescriptionHeightList{{{0.77,0.77},{0.77,0.77,0.77,0.77}}}
\def\DescriptionWidthList{{{5.68,5.68},{5.68,5.68,5.68,5.68}}}
\def\CDaFontSizeList{{"0.45cm","0.45cm",}}
\def\CDbFontSizeList{{"0.39cm","0.39cm",}}
\def\ShowPointingBoxesList{{1,0}}
\def\PointingBoxWidthList{{6,4.5}}
\def\PointingBoxHeightList{{2.5,2.5}}
\def\PointingLetterFontSizeList{{"1.8cm","1.8cm",}}
\def\PointingPhoneticFontSizeList{{"0.6cm","0.6cm",}}
\def\StationIDFontSizeList{{"0.7cm","0.7cm",}}
\def\SheetCheckBoxWidthList{{1.5,1.5}}
\def\SheetCheckBoxHeightList{{1.5,1.5}}
\def\CheckNumberHeightList{{"0.8cm","0.8cm",}}
\def\RemoveTextFontSizeList{{"0.3cm","0.3cm",}}


\begin{document}

%Reduce line spacing. Need to create a TikZ picture first to set pgflinewidth.
\begin{tikzpicture}
\end{tikzpicture}%
\setlength{\lineskip}{-\pgflinewidth}%
%Avoid adding any spaces between TikZ pictures
\newcounter{StationNumber}%
\newcounter{ProblemNumber}%
\forloop{StationNumber}{0}{\value{StationNumber} < \NumStations}{%
\forloop{ProblemNumber}{0}{\not{\value{ProblemNumber} > \MaxProblemsPerStation}}{%
\linebreak[0]%Can break line here without inserting a horizontal space
\begin{tikzpicture}
\pgfmathtruncatemacro{\SheetNumber}{\value{ProblemNumber}}	%Sheet 0 is the cover sheet
\def\StaNumber{\value{StationNumber}}

%Check whether station is hidden
\pgfmathtruncatemacro{\ShowStation}{\ShowStationList[\StaNumber]}
\ifthenelse{\ShowStation = 0}{}{

%Check whether all problems have already been set for this station - the for loop cannot handle a variable ProblemsPerStation
\pgfmathsetmacro{\ProblemsPerStation}{\ProblemsPerStationList[\StaNumber]}
\ifthenelse{\SheetNumber > \ProblemsPerStation}{}{

%Read in arrays of file data
\pgfmathsetmacro{\MapScale}{\MapScaleList[\StaNumber]}
\pgfmathsetmacro{\ContourInterval}{\ContourIntervalList[\StaNumber]}
\pgfmathsetmacro{\CircleRadius}{\CircleRadiusList[\StaNumber]}
\pgfmathsetmacro{\BriefingWidth}{\BriefingWidthList[\StaNumber]}
\pgfmathsetmacro{\SquareMap}{\SquareMapList[\StaNumber]}
\pgfmathsetmacro{\StationID}{\StationIDList[\StaNumber]}
\pgfmathsetmacro{\StationIDFontSize}{\StationIDFontSizeList[\StaNumber]}
\pgfmathtruncatemacro{\NumKites}{\NumKitesList[\StaNumber]}
\pgfmathsetmacro{\ZeroOption}{\ZeroOptionList[\StaNumber]}
\pgfmathsetmacro{\MapHeading}{\MapHeadingList[\StaNumber]}
\pgfmathsetmacro{\CDaFontSize}{\CDaFontSizeList[\StaNumber]}
\pgfmathsetmacro{\CDbFontSize}{\CDbFontSizeList[\StaNumber]}
\pgfmathsetmacro{\PointingLetterFontSize}{\PointingLetterFontSizeList[\StaNumber]}
\pgfmathsetmacro{\PointingPhoneticFontSize}{\PointingPhoneticFontSizeList[\StaNumber]}
\pgfmathsetmacro{\PointingBoxWidth}{\PointingBoxWidthList[\StaNumber]}
\pgfmathsetmacro{\PointingBoxHeight}{\PointingBoxHeightList[\StaNumber]}
\pgfmathsetmacro{\SheetCheckBoxWidth}{\SheetCheckBoxWidthList[\StaNumber]}
\pgfmathsetmacro{\SheetCheckBoxHeight}{\SheetCheckBoxHeightList[\StaNumber]}
\pgfmathsetmacro{\RemoveTextFontSize}{\RemoveTextFontSizeList[\StaNumber]}
\pgfmathsetmacro{\CheckNumberHeight}{\CheckNumberHeightList[\StaNumber]}
\pgfmathtruncatemacro{\ShowPointingBoxes}{\ShowPointingBoxesList[\StaNumber]}
\ifthenelse{\ShowPointingBoxes = 0}{\def\PointingBoxHeight{0}}{}

\ifthenelse{\SheetNumber = 0}
{\pgfmathsetmacro{\DescriptionHeight}{\DescriptionHeightList[\StaNumber][0]}} %Set map position according to size of control descriptions for first problem
{
\pgfmathsetmacro{\MapFile}{\MapFileList[\StaNumber][\SheetNumber-1]}
\pgfmathtruncatemacro{\MapPage}{\MapPageList[\StaNumber][\SheetNumber-1]}
\pgfmathsetmacro{\ControlxCoordinate}{\ControlxCoordinateList[\StaNumber][\SheetNumber-1]}
\pgfmathsetmacro{\ControlyCoordinate}{\ControlyCoordinateList[\StaNumber][\SheetNumber-1]}
\pgfmathsetmacro{\DescriptionFile}{\DescriptionFileList[\StaNumber][\SheetNumber-1]}
\pgfmathtruncatemacro{\DescriptionPage}{\DescriptionPageList[\StaNumber][\SheetNumber-1]}
\pgfmathsetmacro{\DescriptionxCoordinate}{\DescriptionxCoordinateList[\StaNumber][\SheetNumber-1]}
\pgfmathsetmacro{\DescriptionyCoordinate}{\DescriptionyCoordinateList[\StaNumber][\SheetNumber-1]}
\pgfmathsetmacro{\DescriptionHeight}{\DescriptionHeightList[\StaNumber][\SheetNumber-1]}
\pgfmathsetmacro{\DescriptionWidth}{\DescriptionWidthList[\StaNumber][\SheetNumber-1]}}

%Define variables/constants - these are macros, so need brackets when combining
\newcommand{\CardHalfHeight}{6.75}
\newcommand{\CardHalfWidth}{9.5}

%Check for adjust mode
\ifthenelse{\AdjustMode = 1}{\newcommand{\Heading}{0}}{\newcommand{\Heading}{\MapHeading}}

%Pointing boxes
\ifthenelse{\ShowPointingBoxes = 0}{}{
\pgfmathsetmacro{\HalfNumPointingBoxes}{0.5*(\NumKites+\ZeroOption)}
\draw (-\HalfNumPointingBoxes*\PointingBoxWidth,\PointingBoxHeight-\CardHalfHeight) -- (\HalfNumPointingBoxes*\PointingBoxWidth,\PointingBoxHeight-\CardHalfHeight);

\foreach \Boxnumber in {0,...,\NumKites}
{\pgfmathsetmacro{\xCoord}{(\Boxnumber-\HalfNumPointingBoxes)*\PointingBoxWidth}
\draw (\xCoord,-\CardHalfHeight) --  (\xCoord,\PointingBoxHeight-\CardHalfHeight);
%Set labels
\ifthenelse{\Boxnumber = \NumKites}{}{
\ifcase \Boxnumber
\def\BoxLetter{A} \def\BoxPhonetic{Alpha}
\or \def\BoxLetter{B} \def\BoxPhonetic{Bravo}
\or \def\BoxLetter{C} \def\BoxPhonetic{Charlie}
\or \def\BoxLetter{D} \def\BoxPhonetic{Delta}
\or \def\BoxLetter{E} \def\BoxPhonetic{Echo}
\else \def\BoxLetter{F} \def\BoxPhonetic{Foxtrot}
\fi
\node[anchor=south] at (\xCoord+0.5*\PointingBoxWidth,-\CardHalfHeight) {\fontsize{\PointingLetterFontSize}{\PointingLetterFontSize}\selectfont \BoxLetter};
\node[anchor=north] at (\xCoord+0.5*\PointingBoxWidth,\PointingBoxHeight-\CardHalfHeight) {\fontsize{\PointingPhoneticFontSize}{\PointingPhoneticFontSize}\selectfont \BoxPhonetic};
}}

\ifthenelse{\ZeroOption = 1}{
\def\BoxLetter{Z}
\def\BoxPhonetic{Zero}
\pgfmathsetmacro{\xCoord}{(\NumKites-\HalfNumPointingBoxes+1)*\PointingBoxWidth}
\node[anchor=south] at (\xCoord-0.5*\PointingBoxWidth,-\CardHalfHeight) {\fontsize{\PointingLetterFontSize}{\PointingLetterFontSize}\selectfont \BoxLetter};
\node[anchor=north] at (\xCoord-0.5*\PointingBoxWidth,\PointingBoxHeight-\CardHalfHeight) {\fontsize{\PointingPhoneticFontSize}{\PointingPhoneticFontSize}\selectfont \BoxPhonetic};
\draw (\xCoord,-\CardHalfHeight) --  (\xCoord,\PointingBoxHeight-\CardHalfHeight);}{}
}

%Sheet order check boxes - sheet 0 is the cover sheet
\pgfmathsetmacro{\CheckBoxLeftEdge}{\CardHalfWidth-\SheetCheckBoxWidth}
\pgfmathsetmacro{\CheckBoxTopEdge}{0.5*(\PointingBoxHeight+\ProblemsPerStation*\SheetCheckBoxHeight)}

\ifthenelse{\SheetNumber = \ProblemsPerStation}{
\draw (\CheckBoxLeftEdge,\CheckBoxTopEdge-\ProblemsPerStation*\SheetCheckBoxHeight) -- (\CardHalfWidth,\CheckBoxTopEdge-\ProblemsPerStation*\SheetCheckBoxHeight);
}{
\draw[red] (\CheckBoxLeftEdge,\CheckBoxTopEdge-\ProblemsPerStation*\SheetCheckBoxHeight) -- (\CardHalfWidth,\CheckBoxTopEdge-\ProblemsPerStation*\SheetCheckBoxHeight);
}
\foreach \Boxnumber in {1,...,\ProblemsPerStation}
{\pgfmathsetmacro{\LineBottom}{\CheckBoxTopEdge-\Boxnumber*\SheetCheckBoxHeight}

\ifthenelse{\Boxnumber < \SheetNumber}{
%Draw a cross
\draw (\CheckBoxLeftEdge,\LineBottom) -- (\CardHalfWidth,\LineBottom+\SheetCheckBoxHeight);
\draw (\CheckBoxLeftEdge,\LineBottom+\SheetCheckBoxHeight) -- (\CardHalfWidth,\LineBottom);
\draw (\CheckBoxLeftEdge,\LineBottom) -- (\CheckBoxLeftEdge,\LineBottom+\SheetCheckBoxHeight) -- (\CardHalfWidth,\LineBottom+\SheetCheckBoxHeight);
}{
\ifthenelse{\Boxnumber = \SheetNumber}{
\node at (\CheckBoxLeftEdge+0.5*\SheetCheckBoxWidth,\LineBottom+0.5*\SheetCheckBoxHeight) {\resizebox{!}{\CheckNumberHeight}{\Boxnumber}};
\draw (\CheckBoxLeftEdge,\LineBottom) -- (\CheckBoxLeftEdge,\LineBottom+\SheetCheckBoxHeight) -- (\CardHalfWidth,\LineBottom+\SheetCheckBoxHeight);
}{
%Label for removal
\node at (\CheckBoxLeftEdge+0.5*\SheetCheckBoxWidth,\LineBottom+0.5*\SheetCheckBoxHeight) {\fontsize{\RemoveTextFontSize}{\RemoveTextFontSize}\selectfont Remove};
\draw[red] (\CheckBoxLeftEdge,\LineBottom) -- (\CheckBoxLeftEdge,\LineBottom+\SheetCheckBoxHeight) -- (\CardHalfWidth,\LineBottom+\SheetCheckBoxHeight);
}}}

\pgfmathsetmacro{\MapDescriptionSeparation}{1/3*(2*\CardHalfHeight-2*\CircleRadius-\DescriptionHeight-\PointingBoxHeight)} %Amount of vertical white space between top and map, map and descriptions, and descriptions and pointing boxes
\pgfmathsetmacro{\MapCentreyCoordinate}{\CardHalfHeight-\CircleRadius-\MapDescriptionSeparation}
\pgfmathsetmacro{\DescriptionBaseyCoordinate}{-\CardHalfHeight+\PointingBoxHeight+\MapDescriptionSeparation}
\pgfmathsetmacro{\NorthArrowCentrexCoordinate}{-0.5*(\CardHalfWidth+\CircleRadius)}

%Origin is at centre of card - draw bounding lines for cut out
\draw[red] (-\CardHalfWidth,-\CardHalfHeight) rectangle (\CardHalfWidth,\CardHalfHeight);

%Print TCTemplate stamp
\node[anchor=north west] at (-\CardHalfWidth,\CardHalfHeight) {\fontsize{0.35cm}{0.35cm}\selectfont Made in TCTemplate};

%Print station number
\node[anchor=north east] at (\CardHalfWidth,\CardHalfHeight) {\fontsize{\StationIDFontSize}{\StationIDFontSize}\selectfont Station: \StationID{}};

%Map excerpt
\begin{scope}[shift={(0,\MapCentreyCoordinate)}]
\ifthenelse{\SheetNumber = 0}{
\pgfmathtruncatemacro{\TimeLimit}{30*\ProblemsPerStation}
\node at (0,0) {\resizebox{\BriefingWidth}{!}{\parbox{5.1cm}{\centering \fontsize{1cm}{1.2cm}\selectfont \ProblemsPerStation{} tasks \\ Limit: \TimeLimit{}\,s \\ \fontsize{0.8cm}{0.96cm}\selectfont 1:\MapScale \\ \fontsize{0.6cm}{0.72cm}\selectfont \ContourInterval{}\,m contours}}};
}{
\ifthenelse{\SquareMap = 1}{\clip (-\CircleRadius, -\CircleRadius) rectangle (\CircleRadius, \CircleRadius);}{\clip (0,0) circle (\CircleRadius cm);}
{\pgfmathsetmacro{\cliplx}{\ControlxCoordinate-1.5*\CircleRadius}
\pgfmathsetmacro{\cliprx}{\ControlxCoordinate+1.5*\CircleRadius}
\pgfmathsetmacro{\cliply}{\ControlyCoordinate-1.5*\CircleRadius}
\pgfmathsetmacro{\clipry}{\ControlyCoordinate+1.5*\CircleRadius}
\newcommand{\cliplxc}{\cliplx cm}
\newcommand{\cliprxc}{\cliprx cm}
\newcommand{\cliplyc}{\cliply cm}
\newcommand{\clipryc}{\clipry cm}
\node at (0,0) {
\rotatebox{\Heading}{
\includegraphics[page=\MapPage, clip, viewport=\cliplxc{} \cliplyc{} \cliprxc{} \clipryc{}]{\MapFile}
}};}
%Alignment circle
\ifthenelse{\AdjustMode = 1}{\draw[ultra thick, blue] (0,0) circle (0.4 cm);}{}
}
%Border round map
\ifthenelse{\SquareMap = 1}{\draw[ultra thick] (-\CircleRadius, -\CircleRadius) rectangle (\CircleRadius, \CircleRadius);}{\draw[ultra thick] (0,0) circle (\CircleRadius cm);}
\end{scope}

%Control descriptions
\ifthenelse{\SheetNumber = 0}{}{
\begin{scope}[shift={(-0.5*\DescriptionWidth,\DescriptionBaseyCoordinate)}]
\clip (0,0) rectangle (\DescriptionWidth,\DescriptionHeight);
\node[anchor=south west, inner sep=0] at (-\DescriptionxCoordinate,-\DescriptionyCoordinate) {\includegraphics[page=\DescriptionPage]{\DescriptionFile}};
%Problem number
\node[anchor=center, inner sep=0cm, minimum size=0.6cm, fill=white] at (0.07*\DescriptionWidth,0.52*\DescriptionHeight) {\fontseries{b}\fontsize{\CDaFontSize}{\CDaFontSize}\selectfont \SheetNumber{}};
%Number of kites
\ifcase \NumKites
\or \def\ColB{A}
\or \def\ColB{A-B}
\or \def\ColB{A-C}
\or \def\ColB{A-D}
\or \def\ColB{A-E}
\else \def\ColB{A-F}
\fi
\node[anchor=center, inner sep=0cm, minimum size=0.6cm, fill=white] at (0.192*\DescriptionWidth,0.52*\DescriptionHeight) {\fontseries{m}\fontsize{\CDbFontSize}{\CDbFontSize}\selectfont \ColB};
\end{scope}}

%North arrow
\begin{scope}[shift={(\NorthArrowCentrexCoordinate,\MapCentreyCoordinate)}, rotate=\Heading]
\draw[-latex,ultra thick, blue, line width=0.3cm] (0,-1.5) -- (0,1.5);
\node[blue] at (0,-0.5) {\rotatebox{\Heading}{\resizebox{!}{1.5cm}{\Huge N}}};
\end{scope}

}	%End ProblemsPerStation if statement
}	%End ShowStation if statement
\end{tikzpicture}%Comments to ensure no spaces are added between pictures
}}

\end{document}
